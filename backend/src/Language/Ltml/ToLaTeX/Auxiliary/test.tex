\documentclass{article}\usepackage{helvet}\renewcommand{\familydefault}{\sfdefault}\usepackage[letterpaper,top=2cm,bottom=2cm,left=3cm,right=3cm,marginparwidth=1.75cm]{geometry}\usepackage[colorlinks=true,allcolors=red]{hyperref}\usepackage{enumitem}\usepackage{tabularx}\usepackage[T1]{fontenc}\usepackage[utf8]{inputenc}\usepackage{fancyhdr}\usepackage{lastpage}\pagestyle{fancy}\fancyhf{}\setlistdepth{9}
\begin{document}
	\hypertarget{par2}{
	\begin{center}
		\textbf{§ 1\\Studienaufbau}
	\end{center}
}
	\begin{enumerate}[start=1,label=(\arabic*)]
		\item{Die Regelstudienzeit und das Studienvolumen betragen drei Jahre beziehungsweise 180 Leistungspunkte und circa 120 Semesterwochenstunden.}
	\end{enumerate}

	\begin{enumerate}[start=2,label=(\arabic*)]
		\item{Der Bachelorstudiengang Wirtschaftsinformatik (Studienverlaufsplan siehe Anlage {\Large{??}}, Module mit Prüfungsleistungen siehe Anlage {\Large{??}} und {\Large{??}}) setzt sich aus folgenden Bereichen zusammen:
		\begin{enumerate}[label=\arabic*.]
			\item{Grundmodule (Pflicht, BSc-WInf-G) im Umfang von 46 LP:
			\begin{enumerate}[label=\alph*)]
				\item{- infEWInf-01a: Einführung in die Wirtschaftsinformatik (7 LP)}
				\item{- infEInf-01a: Einführung in die Informatik (8 LP)}
				\item{- Inf-InfRecht: Informatikrecht (2 LP)}
			\end{enumerate}
}
			\item{Aufbaumodule (Pflicht, BSc-WInf-A) im Umfang von 32 LP:
			\begin{enumerate}[label=\alph*)]
				\item{- VWL-STATWX: Statistische Methoden (10 LP)}
				\item{- infST-01a: Softwaretechnik (7 LP)}
				\item{- infEthik-01a: Ethik in der Informatik (2 LP)}
			\end{enumerate}
}
			\item{Wahlpflichtmodule (BSc-WInf-WP-WInf und BSc-WInf-WP-Inf) im Umfang von 37 LP. Weitere Informationen sind in Anlage {\Large{??}} aufgeführt. Diese Module bestehen in der Regel aus einer Vorlesung mit einer begleitenden Übung. Bei der Wahl der Module müssen die Studierenden mindestens 23 LP aus dem Bereich Wahlpflichtmodule BSc-WInf-WP-WInf und mindestens 7 LP aus dem Bereich Wahlpflichtmodule BSc-WInf-WP-Inf wählen.}
			\item{ein Projekt Wirtschaftsinformatik (BSc-WInf-Proj) im Umfang von 6 LP.}
			\item{ein Seminarmodul zur Wirtschaftsinformatik (BSc-WInf-Sem) im Umfang von 7 LP, gemäß Anlage {\Large{??}}.}
			\item{Studienangebote BWL oder VWL im Umfang von 40 LP: Die Studienangebote BWL und VWL sind Alternativen. Mit der Wahl des ersten Moduls, welches nicht in beiden Varianten vorkommt, legen Studierende fest, welche Variante sie wählen. Ein Wechsel zu dem jeweils anderen Studienangebot ist jederzeit möglich. Zum Erreichen des Bachelorabschlusses müssen alle Module eines der beiden Studienangebote erfolgreich absolviert werden.
			\begin{enumerate}[label=\alph*)]
				\item{Studienangebot BWL, gemäß Anlage {\Large{??}}:
				\begin{enumerate}[label=\alph*\alph*)]
					\item{BWL-EinfBWL: Einführung in die Betriebswirtschaftslehre (5 LP)}
					\item{BWL-ERW: Externes Rechnungswesen (5 LP)}
					\item{BWL-ER: Entscheidungsrechnungen (5 LP)}
					\item{VWL-EVWL: Einführung in die Volkswirtschaftslehre}
					\item{BWL-InnoMProz: Innovationsmanagement: Prozesse und Methoden (5 LP)}
					\item{Zwei Module des Wahlpflichtbereichs BWL (zusammen 10 LP):
					\begin{enumerate}[label=\alph*\alph*\alph*)]
						\item{- BWL-Ent: Decision Analysis I (5 LP)}
						\item{- BWL-RDM: Decision Analysis II (5 LP)}
						\item{- BWL-ProdLog: Produktion und Logistik (5 LP)}
						\item{- BWL-Mark: Marketing (5 LP)}
					\end{enumerate}
Wurde ein Wahlpflichtmodul erfolgreich abgeschlossen, darf dieses nicht durch ein anderes Wahlpflichtmodul (zum Beispiel zur Notenverbesserung) ersetzt werden.}
				\end{enumerate}
}
				\item{Studienangebot VWL, gemäß Anlage {\Large{??}}:
				\begin{enumerate}[label=\alph*\alph*)]
					\item{BWL-EinfBWL: Einführung in die Betriebswirtschaftslehre (5 LP)}
					\item{VWL-EVWL: Einführung in die Volkswirtschaftslehre (10 LP)}
					\item{VWLvwlMikro1-01a: Grundzüge der mikroökonomischen Theorie I (5 LP)}
				\end{enumerate}
}
			\end{enumerate}
}
			\item{Bachelorarbeit, individuell oder im Abschlussprojekt, im Umfang von 12 LP gemäß § {\Large{??}}.}
		\end{enumerate}
}
	\end{enumerate}

\end{document}
